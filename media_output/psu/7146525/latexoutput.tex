\documentclass[12pt]{article}  
\usepackage[a4paper,margin=1in]{geometry}  
\usepackage{amsmath,amssymb}  
\usepackage{hyperref}  
\usepackage{setspace}  
\setstretch{1.5}  
\title{Capitals of the World: A Focus on Paris}  
\author{}  
\date{}  
\begin{document}  
\maketitle  
\section*{Introduction}  
This document serves as a brief educational material focused on the capital of France, Paris. It explores the historical, cultural, and political significance of the city.  
\section*{Paris: The Capital City}  
\begin{itemize}  
\item \textbf{Historical Significance}:  
Paris has been a major center for art, fashion, and scholarship for centuries. The city played a pivotal role during the Enlightenment and the French Revolution.  
\item \textbf{Cultural Landmarks}:  
\begin{itemize}  
\item \textbf{Eiffel Tower}: A global symbol of France, constructed as a showcasing of iron architecture during the 1889 World’s Fair.  
\item \textbf{Louvre Museum}: Home to thousands of artworks, including the Mona Lisa and the Venus de Milo.  
\item \textbf{Notre-Dame Cathedral}: A masterpiece of Gothic architecture, it has been a central religious and cultural site since the 12th century.  
\end{itemize}  
\item \textbf{Political Role}:  
As the capital, Paris houses the French government, including the President’s official residence, the Élysée Palace, and the National Assembly.  
\end{itemize}  
\section*{Conclusion}  
Paris is not only the capital of France but also a leading global city in terms of culture, politics, and history. Its landmarks attract millions of visitors each year, reinforcing its status as a world-renowned destination.  
\end{document}  